\documentclass{article}
\usepackage{graphicx,fancyhdr,amsmath,amssymb,amsthm,subfig,url,hyperref}
\usepackage[margin=1in]{geometry}

\usepackage{natbib}
\usepackage{cancel}

\usepackage{minted}

\usepackage[shortlabels]{enumitem}

% ========================
%  Khusus Pseudo code
% ========================
\usepackage{algorithm}
\usepackage[noend]{algpseudocode}


%----------------------- Macros and Definitions --------------------------

%%% FILL THIS OUT
\newcommand{\studentname}{Jane Doe}
\newcommand{\suid}{janedoe}
\newcommand{\exerciseset}{Homework}
%%% END



\renewcommand{\theenumi}{\bf \Alph{enumi}}

\theoremstyle{plain}
\newtheorem{theorem}{Theorem}
\newtheorem{lemma}[theorem]{Lemma}


\graphicspath{{figures/}}

%-------------------------------- Title ----------------------------------

\title{Study Kasus 4: \textit{Text Search}} 
\author{IN232 Matematika Diskrit}
\date{}
%--------------------------------- Text ----------------------------------

\begin{document}
\maketitle

%	\begin{enumerate}[-,topsep=0pt, nosep,label=\alph*. ]
%		\item How many ways are there to pick two representatives, so that one is a mathematics major and the other is a computer science major?
%		\item How many ways are there to pick one representative who is either a mathematics major or a computer science major?
%	\end{enumerate}

\section*{Pencarian (\textit{Searching})}
Sebagian besar waktu komputer digunakan untuk pencarian (\textit{searching}). Contoh-contoh pencarian dengan menggunakan komputer, antara lain:
	\begin{enumerate}[-,topsep=0pt, nosep,label=\alph*. ]
		\item mencari \textit{record} di bank untuk seorang teller;
		\item mencari solusi untuk teka-teki atau untuk gerakan yang optimal dalam suatu permainan;
		\item menggunakan mesin pencari (\textit{search engine}) di web;
		\item mencari teks tertentu dalam dokumen saat menjalankan pengolah kata.
	\end{enumerate}

\bigskip
\noindent Misalkan kita diberikan teks dokumen $t$ dan anda ingin menemukan kemunculan pertama dari pola $p$ di $t$ (misalnya, anda ingin menemukan kemunculan pertama dari string $p$ = "Nova Scotia" di $t$) atau menentukan bahwa $p$ tidak ditemukan di $t$. Anda mengindeks karakter dalam $t$ untuk dimulai dari 1. Salah satu pendekatan untuk mencari $p$ adalah dengan memeriksa apakah $p$ muncul pada indeks 1 di $t$. Jika anda menemukan kemunculan pertama $p$ di $t$, anda berhenti. Jika tidak, anda memeriksa apakah $p$ terjadi pada indeks 2 di $t$. Jika $p$ ditemukan di indeks 2, anda berhenti. Jika tidak, selanjutnya anda periksa apakah $p$ terjadi pada indeks 3 di $t$, dan seterusnya.

\begin{algorithm}
\caption{Algoritma \texttt{text\_search}}
\begin{algorithmic}[1] 
\Function{text\_search}{$p$, $m$, $t$, $n$}: 
\newline \Comment{Input: $p$ (berindeks dari 1 sampai $m$), $m$, $t$ (berindeks dari 1 to $n$), $n$}
\newline \Comment{Output: $i$}
\For{$i = 1$ \textbf{to} $n-m+1$} 
	\State $j = 1$
	\newline \Comment{$i$ adalah indeks di dalam $t$ dari karakter pertama dari substring}
	\newline \Comment{untuk membandingkannya dengan $p$, dan $j$ adalah indeks dalam $p$}
	\newline \Comment{the while loop membandingkan $t_i \cdots t_{i+m-1}$ dan $p_1 \cdots p_m$} 
	\While{$t_{i+j-1} == p_j$}
		\State  $j = j + 1$
		\If{$j>m$}
			\State return i
		\EndIf
	\EndWhile
\EndFor
	\State return 0
\EndFunction
\end{algorithmic}
\label{alg:text-search}
\end{algorithm}

\section*{Implementasi}
Tugas anda adalah mengimplementasi Algoritma \ref{alg:text-search}. \\
Buatlah juga minimal 3 \textit{test case} untuk menguji algoritma \textit{text search} anda.

%
%\section*{Implementasi}
%	\begin{enumerate}[-,topsep=0pt, nosep,label=\arabic*. ]
%		\item Ubahlah relasi rekurensi di Persamaan \eqref{eq:price-recursive-relation} menjadi \textit{solusi eksplisit}.  
%		\item Implementasi Persamaan \eqref{eq:price-recursive-relation} dalam program.
%	\end{enumerate}
%
%\begin{minted}{python}
%def compute_cobweb(p_0, a, b, k, n):
%    """
%    Compute a price at n based on cobweb recursive relation	
%    
%    Parameters:
%    -----------
%    p_0 : float
%        An initial price
%    
%    a, b, k : float
%        Three positive parameters
%    
%    n : int
%        The final time for the price to be computed    
%    
%    Returns:
%    --------
%    p_n : float
%        The final price at time n
%    """
%    pass
%\end{minted}
%
%\noindent Sebagai test drive, silakan anda hitung secara manual dan bandingkan hasil manual dengan hasil program.


\bibliographystyle{apalike}
\bibliography{references}


\end{document}
